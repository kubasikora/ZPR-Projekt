\documentclass{article}

% Polskie znaki
\usepackage{polski}
\usepackage[utf8]{inputenc}
\usepackage[T1]{fontenc}
\usepackage{lmodern}
\usepackage{indentfirst}

% Strona tytułowa
\usepackage{pgfplots}
\usepackage{siunitx}
\usepackage{paracol}
\usepackage{gensymb}

% Pływające obrazki
\usepackage{float}
\usepackage{svg}
\usepackage{graphicx}

% table of contents refs
\usepackage{hyperref}
\usepackage{cleveref}
\usepackage{booktabs}
\usepackage{listings}
\usepackage{placeins}
\usepackage{xcolor}

\sisetup{detect-weight,exponent-product=\cdot,output-decimal-marker={,},per-mode=symbol,binary-units=true,range-phrase={-},range-units=single}
\definecolor{szary}{rgb}{0.95,0.95,0.95}
%konfiguracje pakietu listings
\lstset{
	backgroundcolor=\color{szary},
	frame=single,
	breaklines=true,
}
\lstdefinestyle{customlatex}{
	basicstyle=\footnotesize\ttfamily,
	%basicstyle=\small\ttfamily,
}
\lstdefinestyle{customc}{
	breaklines=true,
	frame=tb,
	language=C,
	xleftmargin=0pt,
	showstringspaces=false,
	basicstyle=\small\ttfamily,
	keywordstyle=\bfseries\color{green!40!black},
	commentstyle=\itshape\color{purple!40!black},
	identifierstyle=\color{blue},
	stringstyle=\color{orange},
}
\lstdefinestyle{custommatlab}{
	captionpos=t,
	breaklines=true,
	frame=tb,
	xleftmargin=0pt,
	language=matlab,
	showstringspaces=false,
	%basicstyle=\footnotesize\ttfamily,
	basicstyle=\scriptsize\ttfamily,
	keywordstyle=\bfseries\color{green!40!black},
	commentstyle=\itshape\color{purple!40!black},
	identifierstyle=\color{blue},
	stringstyle=\color{orange},
}

%wymiar tekstu
\def\figurename{Rys.}
\def\tablename{Tab.}

%konfiguracja liczby p�ywaj�cych element�w
\setcounter{topnumber}{0}%2
\setcounter{bottomnumber}{3}%1
\setcounter{totalnumber}{5}%3
\renewcommand{\textfraction}{0.01}%0.2
\renewcommand{\topfraction}{0.95}%0.7
\renewcommand{\bottomfraction}{0.95}%0.3
\renewcommand{\floatpagefraction}{0.35}%0.5

\SendSettingsToPgf
\title{\bf Specyfikacja wstępna - serwer serii danych \vskip 0.1cm}
\author{Julia Kłos \\ Jakub Sikora}
\date{\today}
\pgfplotsset{compat=1.15}	
\begin{document}
\frenchspacing


\makeatletter
\renewcommand{\maketitle}{\begin{titlepage}
		\begin{center}{
				\LARGE {\bf Politechnika Warszawska}}\\
            \vspace{0.4cm}
            \leftskip-0.8cm
            {\LARGE {\bf \mbox{Wydział Elektroniki i Technik Informacyjnych}}}\\
            
            \vspace{2cm}
            \leftskip-1.8cm
			{\bf \Huge \mbox{Zaawansowane programowanie w C++} \vskip 0.1cm}
		\end{center}
		\vspace{0.1cm}

		\begin{center}
			{\bf \LARGE \@title}
		\end{center}

		\vspace{8cm}
		\begin{paracol}{2}
			\addtocontents{toc}{\protect\setcounter{tocdepth}{1}}
			\subsection*{Zespół:}
			\bf{ \Large{ \noindent\@author \par}}
			\addtocontents{toc}{\protect\setcounter{tocdepth}{2}}

			\switchcolumn \addtocontents{toc}{\protect\setcounter{tocdepth}{1}}
			\subsection*{Prowadzący:}
			\bf{\Large{\noindent mgr inż. Konrad \mbox{Grochowski}}}
			\addtocontents{toc}{\protect\setcounter{tocdepth}{2}}

		\end{paracol}
		\vspace*{\stretch{6}}
		\begin{center}
			\bf{\large{Warszawa, \@date\vskip 0.1cm}}
		\end{center}
	\end{titlepage}
}
\makeatother
\maketitle

\section{Temat projektu}
Projekt polega na przygotowaniu serwera, który zbiera dane i prezentuje 
je klientom w formie wykresów.
Serwer powinien wspierać dwa rodzaje klientów - dostarczających dane 
i odbiorców raportów.
Raporty mogą być prezentowane na np. wbudowanym serwerze HTTP.
Dane mogą być dowolne - np. temperatura, czy kurs jakiejś waluty. 
Wykresy powinny wspierać wyświetlenie wielu różnych serii, 
a także różne zakresy czasowe.
Dane powinny być przechowywane w jakiejś bazie danych.

\section{Rozszerzony opis projektu}
Serwer będzie przyjmował dane pochodzące od czterech urządzeń pomiarowych
zasymulowanych cyfrowo. Urządzenia będą komunikowały się poprzez wystawione REST API
za pomocą metody HTTP POST. Wyniki pomiarów będą przechowywane w relacyjnej bazie danych.

Aplikacja będzie prezentowała aktualne pomiary na specjalnie przygotowanym panelu 
operatorskim, otwieranym z poziomu przeglądarki internetowej. Pomiary te będą automatycznie 
odświeżane aby jak najlepiej oddać działanie rzeczywistych grafik inżynierskich. Panel 
będzie dostępny tylko i wyłącznie po poprawnym uwierzytelnieniu.

Panel będzie umożliwiał operatorowi na przeglądanie historii przebiegów danych w 
zadanych przedziałach czasu. Możliwe będzie wyświetlanie kilku serii danych na raz.
Dane będą serializowane z zadaną dokładnością, obliczaną na podstawie rozmiaru okna
przeglądarki.

\section{Lista funkcjonalności}

\begin{itemize}
	\item Zbieranie danych od czujników
	\item Zapis pomiaru do bazy danych
	\item Wyświetlanie i odświeżanie aktualnych wartości zmiennych procesowych
	\item Pobranie danych pomiarowych z bazy i ich serializacja
	\item Generacja wykresów danych z zadaną dokładnością
	\item Możliwość łączenia wykresów
	\item Uwierzytelnianie
	\item Prezentacja informacji o urządzeniu
\end{itemize}

\newpage
\section{Zakładana architektura systemu}
System informatyczny będzie składał się z czterech komponentów:

\begin{itemize}
	\item Serwera panelu operatorskiego (frontend)
	\item Serwera pośredniczącego (backend)
	\item Modułu serializującego
	\item Relacyjnej bazy danych
\end{itemize}

Panel operatorski zostanie napisany w języku Javascript, przy wykorzystaniu
biblioteki React.js. Serwer pośredniczący napiszemy w języku Python przy użyciu
pakietu Flask, z którego będziemy wywoływać metody obsługujące żądania napisane
w języku C++, połączone z serwerem przy wykorzystaniu biblioteki boost::python.
Docelowo, zakładamy relacyjną postać bazy danych, z wykorzystaniem silnika PostgreSQL.

\section{Planowane testy rozwiązania}
W ramach projektu planujemy przygotować szereg testów systemowych i jednostkowych.
Moduł napisany w języku C++ będzie przetestowany za pomocą testów jednostkowych Boost.Test.
Moduł pośredniczący napisany w języku Python przetestujemy za pomocą narzędzia pytest, a panel
operatorski napisany w języku Javascript zostanie przetestowany przy pomocy programu jest.
Dodatkowo, planujemy przeprowadzić automatyczne testy systemowe za pomocą programu Selenium.

\end{document}